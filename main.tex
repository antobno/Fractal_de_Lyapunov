% Compilar en LuaLaTeX
\documentclass{standalone}
\usepackage{luacode}   % Para ejecutar Lua
\usepackage{tikz}      % Para dibujar

\begin{document}

% ----------------------------------------------------------------
% BLOQUE LUACODE: cálculo y dibujo del fractal de Lyapunov
% ----------------------------------------------------------------
\begin{luacode*}
-- Parámetros de la imagen (ajústalos si quieres)
local W, H        = 200, 200   -- anchura y altura en "píxeles"
local a_min, a_max = 2.5, 4.0   -- rango para parámetro A
local b_min, b_max = 2.5, 4.0   -- rango para parámetro B
local warmup      = 100        -- iteraciones de "calentamiento"
local iters       = 400        -- iteraciones para promedio
local sequence_str = "AB"      -- secuencia de mapas (p. ej. "AABABB")
-- Construimos tabla con la secuencia
local sequence = {}
for c in sequence_str:gmatch(".") do sequence[#sequence+1] = c end
local S = #sequence

-- Función que calcula el exponente de Lyapunov para un par (A,B)
local function lyapunov_exponent(A, B)
  local x = 0.5
  local sum = 0
  for n = 1, warmup + iters do
    -- elegimos r = A o B según la secuencia repetida
    local r = (sequence[((n-1) % S) + 1] == "A") and A or B
    x = r * x * (1 - x)
    if n > warmup then
      sum = sum + math.log(math.abs(r * (1 - 2*x)))
    end
  end
  return sum / iters
end

-- Precomputamos todos los exponentes y normalizamos a [0,1]
local Lmin, Lmax = 1e9, -1e9
local Ltab = {}
for j = 1, H do
  local row = {}
  local B = b_min + (j-1)/(H-1)*(b_max - b_min)
  for i = 1, W do
    local A = a_min + (i-1)/(W-1)*(a_max - a_min)
    local L = lyapunov_exponent(A, B)
    row[i] = L
    if L < Lmin then Lmin = L end
    if L > Lmax then Lmax = L end
  end
  Ltab[j] = row
end

-- Dibujamos con TikZ: cada "píxel" es un pequeño rectángulo
tex.sprint("\\begin{tikzpicture}[xscale=0.02,yscale=0.02,line width=0pt]")
for j = 1, H do
  for i = 1, W do
    local L = Ltab[j][i]
    -- normaliza L a [0,1]
    local v = (L - Lmin) / (Lmax - Lmin)
    -- convertimos a porcentaje para mezclar rojo (v%) y azul (100-v%)
    local pct = math.floor(v * 100 + 0.5)
    tex.sprint(string.format(
      "\\fill[red!%d!blue] (%d,%d) rectangle ++(1,1);",
      pct, i-1, j-1
    ))
  end
end
tex.sprint("\\end{tikzpicture}")
\end{luacode*}

\end{document}